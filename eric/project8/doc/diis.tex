%%% Compile me as follows:
%%% latexmk -pdf -pdflatex="pdflatex -shell-escape" diis.tex

\documentclass[10pt,compress,red]{beamer}
\usetheme{Pittsburgh}

\usepackage{amsmath}

\usepackage[T1]{fontenc}
\usepackage[sc]{mathpazo}
\linespread{1.05} % Palatino needs more leading (space between lines)
\usepackage[mathscr]{euscript}

\usepackage{hyperref}
\usepackage[activate={true,nocompatibility},
            final,
            kerning=true
]{microtype}
\usepackage{bm}
\usepackage{braket}
\usepackage[normalem]{ulem}
\usepackage{paralist}
\usepackage{setspace}
\usepackage{xparse}
\usepackage{graphicx}
\usepackage{default}
\usepackage{float}
\usepackage{caption}
\usepackage{subcaption}

%\newcommand{\vect}[1]{
\DeclareDocumentCommand{\vect}{m}{%
        \ensuremath{\boldsymbol{\mathbf{#1}}}%
}%


\usepackage{minted}
\usemintedstyle{default}

\renewcommand{\theFancyVerbLine} {\sffamily
  \textcolor[rgb]{0.5,0.5,0.7}{\scriptsize
    \oldstylenums{\arabic{FancyVerbLine}}}}

\makeatletter
\newcommand\listofframes{\@starttoc{lbf}}
\makeatother

\addtobeamertemplate{frametitle}{}{%
  \addcontentsline{lbf}{section}{\protect\makebox[2em][l]{%
    \protect\usebeamercolor[fg]{structure}\insertframenumber\hfill}%
  \insertframetitle\par}%
}

%%%%%%%%%%%%%%%%%%%%%%%%%%%%%%%%%%%%%%%%%%%%%%%%%%%%%%%%%%%%%%%%%%%%%%%%%%%%%%

\title{Direct Inversion in the Iterative Subspace}
\subtitle{(DIIS)}
\author{Eric Berquist}
\institute{University of Pittsburgh}
\date{October 7th, 2014}

\begin{document}

\frame{
  \titlepage
}

\begin{frame}
  \tableofcontents
  % \listofframes
\end{frame}

\section{Some Theory}

\begin{frame}
  \frametitle{The Self-Consistent Field Procedure}
  \begin{itemize}
    \item Calculate all one- and two-electron integrals.
    \item Generate a suitable start guess for the MO coefficients.
    \item Form the initial density matrix.
    \item Form the Fock matrix as the core (one-electron) integrals +
      the density matrix times the two-electron integrals.
    \item Diagonalize the Fock matrix. The eigenvectors contain the
      new MO coefficients.
    \item Form the new density matrix. If it is sufficiently close to
      the previous density matrix, we are done, otherwise go to step
      4.
  \end{itemize}
\end{frame}

\begin{frame}
  \frametitle{The Self-Consistent Field Procedure}
\begin{align}
  \label{eq:1}
  \vect{S} \vect{L}_{S} &= \vect{L}_{S} \Lambda_{S} \\
  \label{eq:2}
  \vect{S}^{-1/2} &\equiv \vect{L}_{S} \Lambda^{-1/2} \vect{\tilde{L}}_{S} \\
  \label{eq:3}
  \vect{F}_{0}^{'} &\equiv \vect{\tilde{S}}^{-1/2} \vect{H}^{\textrm{core}} \vect{S}^{-1/2} \\
  \label{eq:4}
  \vect{F}_{0}^{'} \vect{C}_{0}^{'} &= \vect{C}_{0}^{'} \epsilon_{0} \\
  \label{eq:5}
  \vect{C}_{0} &= \vect{S}^{-1/2} \vect{C}_{0}^{'} \\
  \label{eq:6}
  (\vect{D}_{0})_{\mu\nu} &= \sum_{m}^{\textrm{occ}} (\vect{C}_{0})_{\mu}^{m} (\vect{C}_{0})_{\nu}^{m} \\
  \label{eq:7}
  E_{\textrm{elec}}^{0} &= \sum_{\mu\nu}^{\textrm{AO}} D_{\mu\nu}^{0} (H_{\mu\nu}^{\textrm{core}} + F_{\mu\nu}) = \mathrm{tr}(\vect{D}^{0}(\vect{H}^{\textrm{core}} + \vect{F}^{0})) \\
  \label{eq:8}
  \vect{F}^{'} &\equiv \vect{\tilde{S}}^{-1/2} \vect{F} \vect{S}^{-1/2}
\end{align}
Rinse and repeat from steps \ref{eq:4}-\ref{eq:8} until your chosen error
metric is acceptable.
\end{frame}

\begin{frame}
  \frametitle{Techniques for SCF Convergence}
  \begin{itemize}
    \item Damping
    \item Level Shifting
    \item \textbf{Extrapolation}
    \item Direct Minimzation
  \end{itemize}
\end{frame}

\begin{frame}
  \frametitle{Some Working Equations}
  \begin{equation}
    \vect{F}' = \sum_{i} c_{i} \vect{F}_{i}
  \end{equation}
  \begin{align}
    \vect{e}' &= \sum_{i} c_{i} \vect{e}_{i} \\
    &\approx \vect{0} \nonumber
  \end{align}
  where
  \begin{align}
    \vect{e}_{i} &\equiv \vect{F}_{i} \vect{P}_{i} \vect{S} - \vect{S} \vect{P}_{i} \vect{F}_{i} \\
    &= [\vect{F}_{i}, \vect{P}_{i}]
  \end{align}
\end{frame}

\begin{frame}
  \frametitle{Derivation Part 1} Suppose we have a vector from
  the \textit{i}th step of an iterative procedure that can be formed
  as the sum of the final quantity plus some error
  \begin{equation}
    \vect{p}_{i} = \vect{p}_{f} + \vect{e}_{i}
  \end{equation}
  and that a good approximation to the final vector is a linear
  combination of the previous guesses
  \begin{equation}
    \vect{p} = \sum_{i}^{m} c_{i} \vect{p}_{i}
  \end{equation}
  where \textit{m} is a fixed integer (defaults: ORCA 5, GAMESS 10,
  Q-Chem 15). Every time there isn't a limit on the sum, assume it's
  \textit{m}.
\end{frame}

\begin{frame}
  \frametitle{Derivation Part 2}
  Make a substitution for \(\vect{p}_{i}\)
  \begin{align}
    \vect{p} &= \sum_{i} c_{i} (\vect{p}_{f} + \vect{e}_{i}) \\
    &= \vect{p}_{f} \sum_{i} c_{i} + \sum_{i} c_{i} \vect{e}_{i}.
  \end{align}
  At convergence, the error must drop to zero, leaving us with
  \begin{align}
    \vect{p} &= \vect{p}_{f} \sum_{i} c_{i} \\
    &= \vect{p}_{f}.
  \end{align}
  So, we must minimize \(\vect{e}^{\prime}\) under the constraint
  \(\sum_{i} c_{i} = 1\).
\end{frame}

\begin{frame}
  \frametitle{Derivation Part 3} Constrained minimization? Lagrange
  multipliers! To minimize the norm of the error
  \begin{equation}
    \Braket{\vect{e}|\vect{e}} = \sum_{ij}^{m} c_{i}^{*} c_{j} \Braket{\vect{e}_{i}|\vect{e}_{j}},
  \end{equation}
  define the Lagrangian
  \begin{equation}
    \mathscr{L} = \vect{c}^{\dagger}\vect{B}\vect{c} - \lambda \left( 1 - \sum_{i}^{m} c_{i} \right)
  \end{equation}
  where
  \begin{equation}
    B_{ij} = \Braket{\vect{e}_{i}|\vect{e}_{j}}.
  \end{equation}
\end{frame}

\begin{frame}
  \frametitle{Derivation Part 4}
  The messy part is minimizing the Lagrangian
  \begin{align}
    \frac{\partial\mathscr{L}}{\partial c_{k}} = 0 &= \sum_{j} c_{j}B_{kj} + \sum_{i} c_{i}B_{ik} - \lambda \\
&= 2 \sum_{i} c_{i} B_{ki} - \lambda,
  \end{align}
  where the 2 can be absorbed into \(\lambda\) and the coefficients
  are assumed to be real. We now need to solve \(m+1\) linear
  equations
  \begin{equation}
    \begin{pmatrix}
      B_{11} & B_{12} & \cdots & B_{1m} & -1 \\
      B_{21} & B_{22} & \cdots & B_{2m} & -1 \\
      \vdots & \vdots & \ddots & \vdots & -1 \\
      B_{m1} & B_{m2} & \cdots & B_{mm} & -1 \\
      -1 & -1 & \cdots & -1 & 0
    \end{pmatrix}
    \begin{pmatrix}
      c_{1} \\ c_{2} \\ \vdots \\ c_{m} \\ \lambda
    \end{pmatrix}
    =
    \begin{pmatrix}
      0 \\ 0 \\ \vdots \\ 0 \\ -1
    \end{pmatrix}
  \end{equation}
\end{frame}

\begin{frame}
  \frametitle{Some More Working Equations}
  All of this results in solving the linear system
  \begin{equation}
    \vect{B} \vect{c} = \vect{z},
  \end{equation}
  where \vect{z} is the ``zero vector''. Finding the \(\{c_{m}\}\) as
  \begin{equation}
    \vect{c} = \vect{B}^{-1} \vect{z}
  \end{equation}
  is done with a math library, typically involving a call to the
  LAPACK routine \texttt{DGESV}.
\end{frame}

\section{Some Examples}

\begin{frame}
  \frametitle{A 1-by-1 Example}

  \begin{equation}
    \begin{pmatrix}
      B_{11} & -1 \\
      -1 & 0
    \end{pmatrix}
    \begin{pmatrix}
      c_{1} \\ \lambda
    \end{pmatrix}
    =
    \begin{pmatrix}
      0 \\ -1
    \end{pmatrix}
  \end{equation}

  \begin{equation}
    \begin{pmatrix}
      c_{1} \\ \lambda
    \end{pmatrix}
    =
    \begin{pmatrix}
      0 & -1 \\
      -1 & -B_{11}
    \end{pmatrix}
    \begin{pmatrix}
      0 \\ -1
    \end{pmatrix}
  \end{equation}
  Surprise, \(c_{1}\) is 1!
\end{frame}

\begin{frame}
  \frametitle{A 2-by-2 Example}

  \begin{equation}
    \begin{pmatrix}
      B_{11} & B_{12} & -1 \\
      B_{21} & B_{22} & -1 \\
      -1 & -1 & 0
    \end{pmatrix}
    \begin{pmatrix}
      c_{1} \\ c_{2} \\ \lambda
    \end{pmatrix}
    =
    \begin{pmatrix}
      0 \\ 0 \\ -1
    \end{pmatrix}
  \end{equation}

  Results in the following set of equations:

  \begin{align}
    c_{1} B_{11} + c_{2} B_{12} - \lambda &= 0 \\
    c_{1} B_{21} + c_{2} B_{22} - \lambda &= 0 \\
    - c_{1} - c_{2} &= -1
  \end{align}

\end{frame}

\section{Some Code}

\begin{frame}
  \frametitle{Outline of the algorithm}
  \begin{itemize}
    \item Compute the Error Matrix in Each Iteration
    \item Build the B Matrix and Solve the Linear Equations
    \item Compute the New Fock Matrix
  \end{itemize}
\end{frame}

\begin{frame}[fragile]
\frametitle{Computing the Error Matrix}
\begin{minted}[linenos,gobble=0,mathescape]{c++}
/*!
 * @brief Build the DIIS error matrix.
 *
 * The formula for the error matrix at the ith iteration is:
 *  $e_{i}=F_{i}D_{i}S-SD_{i}F_{i}$
 */
arma::mat build_error_matrix(const arma::mat &F,
                             const arma::mat &D,
                             const arma::mat &S) {
  return (F*D*S) - (S*D*F);
}
\end{minted}
\end{frame}

\begin{frame}[fragile]
\frametitle{Building the B Matrix}
\begin{minted}[linenos,gobble=0,mathescape]{c++}
/*!
 * @brief Build the DIIS B matrix, or ``$A$'' in $Ax=b$.
 */
arma::mat build_B_matrix(const deque< arma::mat > &e) {
  int NErr = e.size();
  arma::mat B(NErr + 1, NErr + 1);
  B(NErr, NErr) = 0.0;
  for (int a = 0; a < NErr; a++) {
    B(a, NErr) = B(NErr, a) = -1.0;
    for (int b = 0; b < a + 1; b++)
      B(a, b) = B(b, a) = arma::dot(e[a].t(),  e[b]);
  }
  return B;
}
\end{minted}
\end{frame}

\begin{frame}[fragile]
\frametitle{Computing the New Fock Matrix}
\begin{minted}[linenos,gobble=0,mathescape,fontsize=\small]{c++}
/*!
 * @brief Build the extrapolated Fock matrix from the Fock vector.
 *
 * The formula for the extrapolated Fock matrix is:
 *  $F^{\prime}=\sum_{k}^{m}c_{k}F_{k}$
 * where there are $m$ elements in the Fock and error vectors.
 */
void build_extrap_fock(arma::mat &F_extrap,
                       const arma::vec &diis_coeffs,
                       const deque< arma::mat > &diis_fock_vec) {
  const int len = diis_coeffs.n_elem - 1;
  F_extrap.zeros();
  for (int i = 0; i < len; i++)
    F_extrap += (diis_coeffs(i) * diis_fock_vec[i]);
}
\end{minted}
\end{frame}

\begin{frame}[fragile]
\frametitle{Algorithm: Form RHS}
\begin{minted}[linenos,gobble=0,mathescape]{c++}
/*!
 * @brief Build the DIIS "zero" vector, or "$b$" in $Ax=b$.
 */
arma::vec build_diis_zero_vec(const int len) {
  arma::vec diis_zero_vec(len, arma::fill::zeros);
  diis_zero_vec(len - 1) = -1.0;
  return diis_zero_vec;
}
\end{minted}
\end{frame}

\begin{frame}[fragile]
\frametitle{Algorithm: Data Structures}
\begin{minted}[linenos,gobble=2]{c++}
  /*!
   * Prepare structures necessary for DIIS extrapolation.
   */
  int NErr;
  deque< arma::mat > diis_error_vec;
  deque< arma::mat > diis_fock_vec;
  int max_diis_length = 6;
  arma::mat diis_error_mat;
  arma::vec diis_zero_vec;
  arma::mat B;
  arma::vec diis_coeff_vec;
\end{minted}
\end{frame}

\begin{frame}[fragile]
\frametitle{Algorithm: Main Loop}
\begin{minted}[linenos,gobble=4,fontsize=\footnotesize]{c++}
    // Start collecting elements for DIIS once we're past the first iteration.
    if (iter > 0) {
      diis_error_mat = build_error_matrix(F, D, S);
      NErr = diis_error_vec.size();
      if (NErr >= max_diis_length) {
        diis_error_vec.pop_back();
        diis_fock_vec.pop_back();
      }
      diis_error_vec.push_front(diis_error_mat);
      diis_fock_vec.push_front(F);
      NErr = diis_error_vec.size();
      // Perform DIIS extrapolation only if we have 2 or more points.
      if (NErr >= 2) {
        diis_zero_vec = build_diis_zero_vec(NErr + 1);
        B = build_B_matrix(diis_error_vec);
        diis_coeff_vec = arma::solve(B, diis_zero_vec);
        build_extrap_fock(F, diis_coeff_vec, diis_fock_vec);
      }
    }
\end{minted}
\end{frame}

\begin{frame}[fragile]
  \frametitle{Results: DIIS Off}
  Water, RHF/STO-3G (7 basis functions)
\begin{Verbatim}[fontsize=\tiny]
   0    -117.839710375889    -117.839710375889
   1     -70.284216222929    -117.839710375889
   2     -76.045949191625      47.555494152959       1.826673084479
   3     -74.714598899044      -5.761732968696       0.479570364860
   4     -74.984773695191       1.331350292581       0.086831688906
   5     -74.935766125703      -0.270174796147       0.031026136359
   6     -74.943904016889       0.049007569488       0.010799283179
   7     -74.942119723104      -0.008137891186       0.005254826831
   8     -74.942250190433       0.001784293785       0.002438579642
   9     -74.942136438892      -0.000130467329       0.001177279531
  10     -74.942111868815       0.000113751541       0.000564543180
...
  23     -74.942079930560       0.000000005264       0.000000043256
  24     -74.942079929335       0.000000002540       0.000000020871
  25     -74.942079928743       0.000000001226       0.000000010070
  26     -74.942079928458       0.000000000591       0.000000004859
  27     -74.942079928320       0.000000000285       0.000000002345
  28     -74.942079928254       0.000000000138       0.000000001131
  29     -74.942079928222       0.000000000066       0.000000000546
  30     -74.942079928206       0.000000000032       0.000000000263
  31     -74.942079928199       0.000000000016       0.000000000127
  32     -74.942079928195       0.000000000007       0.000000000061
  33     -74.942079928193       0.000000000004       0.000000000030
  34     -74.942079928193       0.000000000002       0.000000000014
  35     -74.942079928192       0.000000000001       0.000000000007
  36     -74.942079928192       0.000000000000       0.000000000003
  37     -74.942079928192       0.000000000000       0.000000000002
  38     -74.942079928192       0.000000000000       0.000000000001
  39     -74.942079928192       0.000000000000       0.000000000000
  40     -74.942079928192       0.000000000000       0.000000000000
\end{Verbatim}
\end{frame}

\begin{frame}[fragile]
  \frametitle{Results: DIIS On}
  Water, RHF/STO-3G (7 basis functions), 6 error vectors
\begin{Verbatim}[fontsize=\tiny]
   0    -117.839710375889    -117.839710375889
   1     -70.284216222929    -117.839710375889
   2     -74.576672718926      47.555494152959       1.826673084479
   3     -75.105709804074      -4.292456495997       0.403889812696
   4     -74.954655933663      -0.529037085148       0.088003715430
   5     -74.938944396782       0.151053870411       0.020519848928
   6     -74.942105934721       0.015711536881       0.012108640030
   7     -74.942079965219      -0.003161537938       0.000460862177
   8     -74.942079928865       0.000025969502       0.000001081074
   9     -74.942079928192       0.000000036354       0.000000052668
  10     -74.942079928192       0.000000000673       0.000000000336
\end{Verbatim}
\end{frame}

\begin{frame}[fragile]
  \frametitle{DIIS Error Matrix}
After iteration 0:
\begin{Verbatim}[fontsize=\tiny]
                1           2           3           4           5           6           7
    1  -0.0000000  -0.0135594  -0.0000000  -0.0102014   0.0000000  -0.0646745  -0.0646745
    2   0.0135594   0.0000000  -0.0000000  -0.0072338   0.0000000   0.1917308   0.1917308
    3   0.0000000   0.0000000   0.0000000   0.0000000  -0.0000000   0.2279110  -0.2279110
    4   0.0102014   0.0072338  -0.0000000   0.0000000   0.0000000   0.1787515   0.1787515
    5  -0.0000000  -0.0000000   0.0000000  -0.0000000   0.0000000   0.0000000  -0.0000000
    6   0.0646745  -0.1917308  -0.2279110  -0.1787515  -0.0000000   0.0000000   0.0000000
    7   0.0646745  -0.1917308   0.2279110  -0.1787515   0.0000000  -0.0000000   0.0000000
\end{Verbatim}
After iteration 2:
\begin{Verbatim}[fontsize=\tiny]
                1           2           3           4           5           6           7
    1   0.0000000  -0.0001491  -0.0000000   0.0005689  -0.0000000  -0.0004437  -0.0004437
    2   0.0001491   0.0000000  -0.0000000  -0.0156002  -0.0000000   0.0133253   0.0133253
    3   0.0000000   0.0000000   0.0000000   0.0000000  -0.0000000   0.0010536  -0.0010536
    4  -0.0005689   0.0156002  -0.0000000  -0.0000000  -0.0000000  -0.0044211  -0.0044211
    5   0.0000000   0.0000000   0.0000000   0.0000000   0.0000000   0.0000000   0.0000000
    6   0.0004437  -0.0133253  -0.0010536   0.0044211  -0.0000000   0.0000000   0.0000000
    7   0.0004437  -0.0133253   0.0010536   0.0044211  -0.0000000   0.0000000   0.0000000
\end{Verbatim}
\end{frame}

\begin{frame}[fragile]
  \frametitle{DIIS B matrix and coefficients}
After two iterations:
\begin{Verbatim}[fontsize=\tiny]
                1           2           3
    1  -0.0850416   0.1611509  -1.0000000  0.7287
    2   0.1611509  -0.5000366  -1.0000000  0.2713
    3  -1.0000000  -1.0000000   0.0000000 -0.0182
\end{Verbatim}
Just before convergence:
\begin{Verbatim}[fontsize=\tiny]
                1           2           3           4           5           6           7
    1  -0.0000000   0.0000000   0.0000000   0.0000000  -0.0000000   0.0000000  -1.0000000    1.9028e+00
    2   0.0000000  -0.0000000  -0.0000000  -0.0000000   0.0000000  -0.0000000  -1.0000000   -9.0514e-01
    3   0.0000000  -0.0000000  -0.0000000  -0.0000000   0.0000000   0.0000000  -1.0000000    2.0913e-03
    4   0.0000000  -0.0000000  -0.0000000  -0.0000000   0.0000000  -0.0000000  -1.0000000    2.2118e-04
    5  -0.0000000   0.0000000   0.0000000   0.0000000  -0.0000001  -0.0000005  -1.0000000    3.1371e-07
    6   0.0000000  -0.0000000   0.0000000  -0.0000000  -0.0000005  -0.0000142  -1.0000000    8.1938e-09
    7  -1.0000000  -1.0000000  -1.0000000  -1.0000000  -1.0000000  -1.0000000   0.0000000   -1.2659e-27
\end{Verbatim}
\end{frame}

\begin{frame}
  \frametitle{GDIIS}
  In the case of a nearly quadratic energy function,
  \begin{align}
    \vect{e}_{i} &= -\vect{H}^{-1} \vect{g}_{i} \\
    \vect{x}_{m+1}^{\prime} &= \sum_{i} c_{i} \vect{x}_{i} \\
    \vect{g}_{m+1}^{\prime} &= \sum_{i} c_{i} \vect{g}_{i} \\
    \vect{x}_{m+1} &= \vect{x}_{m+1}^{\prime} - \vect{H}^{-1} \vect{g}_{m+1}^{\prime}
  \end{align}
\end{frame}

\begin{frame}
  \frametitle{References}
  \begin{itemize}
  \item
    \url{http://sirius.chem.vt.edu/wiki/doku.php?id=crawdad:programming:project8}
    (T. Daniel Crawford)
  \item Pulay, J Comp Chem, 3 (1982) 556.
  \item \url{http://vergil.chemistry.gatech.edu/notes/diis/diis.html}
    (C. David Sherrill)
  \item Introduction to Computational Chemistry (2nd Ed.), Frank Jensen
  \item \url{http://en.cppreference.com/w/cpp/container/deque}
  \item Q-Chem 4.2 manual
  \end{itemize}
\end{frame}

\end{document}
